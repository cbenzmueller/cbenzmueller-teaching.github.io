\documentclass{letter}
\usepackage{hyperref}

%%%%%%%%%% Start TeXmacs macros
\newcommand{\subsubsection}[1]{\medskip\bigskip

\noindent\textbf{\large #1}}
\newcommand{\tmstrong}[1]{\textbf{#1}}
\newcommand{\tmtextbf}[1]{{\bfseries{#1}}}
\newcommand{\tmtextit}[1]{{\itshape{#1}}}
\newenvironment{itemizedot}{\begin{itemize} \renewcommand{\labelitemi}{$\bullet$}\renewcommand{\labelitemii}{$\bullet$}\renewcommand{\labelitemiii}{$\bullet$}\renewcommand{\labelitemiv}{$\bullet$}}{\end{itemize}}
%%%%%%%%%% End TeXmacs macros

\begin{document}

{\small \begin{center}
  {\large \tmtextbf{Teaching Activities}}
\end{center}

{\noindent}\begin{tabular}{l}
  \hline
  \quad
\end{tabular}

{\tmstrong{Teaching at FU Berlin}}
\begin{itemizedot}
  \item WS 2012/2013{\tmstrong{}}, Course:
  {\tmstrong{}}\href{https://www.mi.fu-berlin.de/kvv/course.htm?cid=10691\&sid=26}{Logik
  erster Stufe in Theorie und Praxis}, 2 SWS, 5 ECTS
  
  \item WS 2012/2013, Seminar:
  \href{https://www.mi.fu-berlin.de/kvv/course.htm?cid=10888\&sid=26\&iid=1}{Technologien
  f\"ur Frage-Antwort-Systeme}, 2 SWS, 4 ECTS (with R. Rojas, M.
  Block-Berlitz) 
\end{itemizedot}
{\noindent}\begin{tabular}{l}
  \hline
  \quad
\end{tabular}

\subsubsection{\label{auto-1}Lecture Courses {\tmstrong{}}in the
Past}\label{auto-1}
\begin{itemize}
  \item Lecture Course in Autumn 2008 at Saarland University:
  \href{http://www.cs.miami.edu/\~{ }geoff/Courses/TPTPSYS/}{Working with
  Automated Reasoning Tools}. (with G. Sutcliffe)
  
  \item Lecture course in Autumn 2007 at IT University in Copenhagen:
  Semantics of Higher-Order Logic (SEMHOL).
  (\href{../papers/2007-Copenhagen-lectures.pdf}{pdf})
  
  \item Lecture course in Summer 2006 at
  \href{http://esslli2006.lcc.uma.es/}{ESSLI 2006} in Malaga: Semantics of
  Higher-Order Logic (SEMHOL). (with C. {\tmstrong{}}Brown)
  (\href{../papers/2006-ESSLLI.pdf}{pdf},
  \href{../papers/2006-ESSLLI-6on1.pdf}{pdf-4on1})
  
  \item Lecture course in Summer 2006 at the University of Darmstadt:
  Automated Theorem Proving in Higher-Order Logics (ATPHOL), 3 SWS +
  exercises. (\href{../papers/2006-ATPHOL.pdf}{pdf},
  \href{../papers/2006-ATPHOL-6on1.pdf}{pdf-4on1})
  
  \item Lecture course in Winter 2005 at the {\tmstrong{}}Saarland University:
  Semantics and Mechanization of Classical Higher-Order Logic (SEMHOL), 4 SWS
  + theoretical and practical exercises. (with C. Brown)
  
  \item Lecture course in Summer 2005 at Saarland University:
  \href{http://www.ags.uni-sb.de/\~{
  }omega/teach/KI05/}{\tmtextit{Introduction to Artificial Intelligence
  (AI)}}, 4 SWS + theoretical and practical exercises. (with J. Siekmann, S.
  Autexier)\tmtextit{.}
  
  \item Lecture course in Winter 2004 at Saarland University:
  \tmtextit{\href{http://www.ags.uni-sb.de/\~{
  }omega/teach/MAS0405/index.php}{Mathematical Assistance Systems (MAS)}}, 4
  SWS + theoretical and practical exercises. (J. Siekmann, S. Autexier, C.
  Brown, A. Fiedler, C.-P. Wirth)
  
  \item Lecture course in Summer 2004 at Saarland University:
  \tmtextit{\href{../lectures/fol-hol-tp/index.html}{Automated Theorem Proving
  in First-Order and Higher-Order Logic.}} 2 SWS + theoretical and practical
  exercises.
  
  \item Lecture course in Winter 2003 at Saarland University:
  \tmtextit{Human-Oriented Theorem Proving}, 4 SWS + theoretical and practical
  exercises. (jointly with C.-P. Wirth and A. Fiedler)
  
  \item Lecture course in Summer 2003 at Saarland University:
  \tmtextit{Introduction to Artificial Intelligence}, 4 SWS + theoretical and
  practical exercises. (with J. Siekmann, E. Melis)\tmtextit{.}
  
  \item Lecture course at CALCULEMUS Autumn School 2002 in Pisa: From Natural
  Deduction Calculus to Sequent Calculus and back.
  
  \item Lecture course in 2002 at Saarland University: \tmtextit{Automated
  Theorem Proving in First-Order and Higher-Order Logic}, 2 SWS + theoretical
  and practical exercises.
  
  \item Lecture course in 2001 at Saarland University: \tmtextit{Introduction
  to Artificial Intelligence} (with J. Siekmann), 4 SWS + theoretical and
  practical exercises.
  
  \item Lecture course in 1999 at Saarland University: \tmtextit{Introduction
  to Artificial Intelligence}(with J. Siekmann), 4 SWS + theoretical and
  practical exercises.
\end{itemize}
{\noindent}\begin{tabular}{l}
  \hline
  \quad
\end{tabular}

\subsubsection{\label{auto-2}Seminars in the Past}\label{auto-2}
\begin{itemize}
  \item Summer 2006: \tmtextit{\href{http://www.ags.uni-sb.de/\~{
  }omega/teach/MASS06/}{Mathematical Assistant System Shootout}}; full
  organisation.
  
  \item Summer 2005: \tmtextit{Proof Assistants Shootout}; full organisation.
  
  \item Winter 2000/2001: \tmtextit{Tutorial Systems}; supervision of
  students.
  
  \item Summer 2000: \tmtextit{Deduction and Computation}; full organisation.
  
  \item Since 2000: weekly or two-weekly OMEGA seminar of the AG Siekmann.
  
  \item Winter 1999/2000:\tmtextit{AI Planning}; {\tmstrong{}}supervision of
  students.
  
  \item Winter 1999/2000:\tmtextit{Deduction Systems}; supervision of
  students.
  
  \item Summer 1998: \tmtextit{Deduction Systems}; supervision of students.
\end{itemize}
}

\end{document}
