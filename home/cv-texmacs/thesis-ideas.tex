\documentclass{article}

\title{Topics for Thesis Projects} 
\author{Christoph Benzm\"uller}

\begin{document}
\maketitle

\section{Fast Clause Normalization for Higher-Order  Automated Theorem Proving}
The task consist in extending the techniques for fast first-order
clause normalization, as, for example, employed in Flotter
\cite{DBLP:conf/cade/WeidenbachGR96} to higher-order logic
\cite{Andrews86} and to make them available in the LEO-II theorem
prover \cite{C26}. This work is theoretically and practically demanding.

\section{Agent-Based Architecture for Cooperative Automated Theorem Proving in Higher-Order and First-Order Logic}
The old LEO-I system \cite{C3} was prototypically integrated with
first-order automated theorem provers using the agent based OANTS
architecture \cite{C6,C7,C8}. This integration has been shown superior
over pure first-order automated theorem proving in selected problem
domains \cite{J16}.

The task is to develop a new and ideally improved OANTS architecture
for the new LEO-II prover \cite{C26}, which is currently using a
primitive, sequential interaction model. The new agent based solution
should ideally support and exploit incremental first-order theorem
provers (e.g., theorem prover E \cite{Sch02-AICOMM} has such an incremental mode).
Moreover, the integration should support semantic brokering of reasoning systems in TPTP (www.tptp.org)
based on ideas as also exploited in the MathServe system \cite{DBLP:conf/cade/ZimmerA06}.
This task is practically very demanding and also theoretically challenging.

A reduced task is to simply concentrate on the (redevelopment of the) OANTS architecture.


\section{Homogeneous Proof Representation for Cooperative Higher-Order-First-Order Resolution Proofs}
LEO-II cooperates with first-order automated theorem provers
\cite{C26}. The first-order provers typically refute a subset of
LEO-II's dynamically growing clause set. Currently the proof output of
LEO-II does not provide detailed information on the proof steps
performed by the first-order provers but instead handles this part of
the proof simply as a black box. The task is to develop a tool (inside or independent
of LEO-II) that generates detailled proof output for such cooperative higher-order-first-order
proofs. The work needs to be carried out in close collaboration with the TPTP project (www.tptp.org)
and should exploit and propose new TPTP proof representation standards. 


\section{A Proof Verifier for Higher-Order Resolution Proofs}
The task is to develop an independent proof verifier for higher-order
resolution proofs that are represented in the new TPTP higher-order
proof representation language -- this is also the native proof
language of LEO-II. The techniques exploited may be freely
chosen/proposed by the student.  E.g., they may consist on combinatory
logic and exploit first-order reasoners or they may be based on a
small, trusted higher-order calulus. In any case, the chosen solution should make
use of the fact that only small steps need to be verified one by one.
This task is theoretically and practically demanding.


\section{A Resolution based Theorem  Prover for the Basic Fragment of  Simple Type Theory}
The basic fragment of simple type theory is simple type theory
\cite{Andrews86} for the single base type $o$ of Booleans. This
fragment is decidable.  The question is whether this fragment can be
efficiently automated exploiting the ideas underlying LEO-II's
extansional higher-order resolution calculus \cite{T2}. Ideally, a
decidability result can be given and a fast implementation within
LEO-II can be realized.  This task is theoretically and practically demanding.


\section{A Web Service to Search Data in TPTP Format (Frank Theiss)}
The web service will exploit CompleteSearch (MPII,
http://search.mpi-inf.mpg.de/) with its features for auto completion
and prefix search to find a suitable data model for a query to the
TPTP library (www.tptp.org), covering both formal data (the formulae)
and informal data (the comments).  The task includes to find a
reasonable set of prefixes for structured search (e.g. symbol:union,
logic:thf, author:sutcliffe) and to implement a toolchain to read in
TPTP files and build indexes for the CompleteSearch
server. Furthermore, the task involves the implemention/adaption of
the search engines user interface (PHP/HTML/javascript).



\bibliographystyle{plain}
\bibliography{thesis-ideas}


\end{document}