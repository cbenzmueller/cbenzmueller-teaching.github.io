\documentclass{article}
\usepackage{hyperref}

%%%%%%%%%% Start TeXmacs macros
\newcommand{\tmabbr}[1]{#1}
\newcommand{\tmtextbf}[1]{{\bfseries{#1}}}
\newcommand{\tmtextsf}[1]{{\sffamily{#1}}}
\newenvironment{itemizedot}{\begin{itemize} \renewcommand{\labelitemi}{$\bullet$}\renewcommand{\labelitemii}{$\bullet$}\renewcommand{\labelitemiii}{$\bullet$}\renewcommand{\labelitemiv}{$\bullet$}}{\end{itemize}}
\newenvironment{itemizeminus}{\begin{itemize} \renewcommand{\labelitemi}{$-$}\renewcommand{\labelitemii}{$-$}\renewcommand{\labelitemiii}{$-$}\renewcommand{\labelitemiv}{$-$}}{\end{itemize}}
%%%%%%%%%% End TeXmacs macros

\begin{document}

\tmtextsf{{\small \begin{center}
  {\large \tmtextbf{Research Profile and Selected Publications}}
\end{center}



Below I first address my research interests in \tmtextbf{artificial
intelligence} and then those in \tmtextbf{(theoretical) computer science}. The
selected order of research topics does not express any personal precedence.



{\noindent}\begin{tabular}{l}
  \hline
  \quad
\end{tabular}

\tmtextbf{\begin{center}
  Artificial Intelligence
\end{center}}



\tmtextbf{Automated Theorem Proving:} Automated theorem proving is the proving
of mathematical theorems by a computer program. A simpler, but related problem
is proof verification, where an existing proof for a theorem is certified
valid. For this, it is generally required that each individual proof step can
be verified by a computer program.
\begin{itemizedot}
  \item Own Research: Development of the automated higher-order theorem prover
  LEO and its integration with state-of-the-art first-order theorem provers
  (e.g. Vampire) based on the multi-agent architecture OANTS (see below).
  
  \item Selected Publications:
  \begin{itemizeminus}
    \item {\small [C16] C. Benzm\"uller, V. Sorge, M. Jamnik, and M. Kerber,
    \tmtextbf{Can a Higher-Order and a First-Order Theorem Prover Cooperate?}}
    {\small In F. Baader, A. Voronkov (eds.){\small , Proceedings of the 11th
    International Conference on Logic for Programming Artificial Intelligence
    and Reasoning (LPAR), \href{http://www.springer.de/comp/lncs/}{LNAI} vol.
    3452, pp. 415-431, \ Montevideo, Uruguay, 2005. Springer.}}
    
    \item {\small [C3] C. Benzm\"uller and M. Kohlhase, \tmtextbf{LEO -- A
    Higher-Order Theorem Prover. }}{\small In}{\small  C. Kirchner and H.
    Kirchner (eds.)}{\small , Proceedings of the 15th International Conference
    on Automated Deduction (CADE),
    \href{http://www.springer.de/comp/lncs/}{LNAI} vol. 1421, pp. 139-143,
    Lindau, Germany, 1998. Springer.}
  \end{itemizeminus}
\end{itemizedot}


\tmtextbf{Knowledge Representation and Reasoning:} Knowledge representation is
needed for processing concepts in an information system. In the field of
artificial intelligence (and also in formal mathematics), problem solving can
be simplified by an appropriate choice of knowledge representation.
Representing the knowledge in one way may make the solution simple, while an
unfortunate choice of representation may make the solution difficult or
obscure.
\begin{itemizedot}
  \item Own Research: Development of the mathematics assistance environment
  OMEGA in the SFB378 at Saarland University. In OMEGA we combine a
  higher-order logic based on Church's simple type theory with the paradigm of
  knowledge based proof planning and deep inference in order to support
  expressive knowledge representation and efficient problem solving in
  mathematics. Furthermore, OMEGA integrates many external reasoning systems.
  Since these systems often use different knowledge representations an
  inportant task is to translate forth and back between them.
  
  \item Selected Publications:
  \begin{itemizeminus}
    \item [J12] J. Siekmann, C. Benzm\"uller, and S. Autexier, \
    \tmtextbf{Computer Supported Mathematics with OMEGA.} Special Issue on
    Mathematics Assistance Systems,
    \href{http://www.elsevier.com/wps/find/journaldescription.cws\_home/672712/description\#description}{Journal
    of Applied Logic}. Elsevier. To appear, 2006.
    
    \item {\small [C15] J. Siekmann and C. Benzm\"uller, \tmtextbf{OMEGA:
    Computer Supported Mathematics.}} In S. Biundo, T. Fr\"uhwirth, and G.
    Palm (eds.), KI 2004: Advances in Artificial Intelligence: 27th Annual
    German Conference on AI, \href{http://www.springer.de/comp/lncs/}{LNAI}
    vol. 3228, pp. 3-28, Ulm, Germany, 2004. Springer.
  \end{itemizeminus}
\end{itemizedot}


\tmtextbf{\tmtextbf{Cognitive Architectures} and Multi-Agent Systems:} A
cognitive architecture proposes (artificial) computational processes that act
like certain cognitive systems, most often, like a person, or acts intelligent
under some definition. It is a superset of general agent architectures. The
term architecture implies an approach that attempts to model not only
behavior, but also structural properties of the modelled system. A multi-agent
system is a system composed of several agents, capable of reaching goals that
are difficult to achieve by an individual system. Multi-agent systems can
manifest self-organization and complex behaviors even when the individual
strategies of all their agents are simple.
\begin{itemizedot}
  \item Own Research: Development of the multi-agent architecture OANTS and
  its exploitation as a base architecture supporting the flexible integration
  of external systems, resource bounded distributed reasoning, and user
  interaction in the mathematics assistance system OMEGA and the higher-order
  theorem prover LEO. \
  
  \item Selected Publications:
  \begin{itemizeminus}
    \item {\small [C10] C. Benzm\"uller, M. Jamnik, M. Kerber, and V. Sorge,
    \tmtextbf{Experiments with an Agent-oriented Reasoning System.}} {\small
    In F. Baader, G. Brewka, and Th. Eiter (eds.), Proceedings of \ KI/OGAI:
    Advances in Artificial Intelligence, Joint German/Austrian Conference on
    AI, \href{http://www.springer.de/comp/lncs/}{LNAI} vol. 2174, pp. 409-424,
    Vienna, Austria, 2001. Springer.}
    
    \item {\small [C8] C. Benzm\"uller and V. Sorge, \tmtextbf{O-ANTS -- An
    open approach at combining Interactive and Automated Theorem Proving.}}
    {\small In M. Kerber and M. Kohlhase (eds.), Integration of \ Symbolic
    Computation and Mechanized Reasoning, pp. 81-97, 2000.
    \href{http://www.akpeters.com/}{A.K.Peters}.}
    
    \item {\small [C6] C. Benzm\"uller and V. Sorge, \tmtextbf{Critical Agents
    Supporting Interactive Theorem Proving. }}{\small In P. Borahona and J. J.
    Alferes (eds.), Proceedings of the 9th Portuguese Conference on Artificial
    Intelligence (EPIA), \href{http://www.springer.de/comp/lncs/}{LNAI} vol.
    1695, pp. 208-221, Evora, Portugal, 1999. Springer.}
  \end{itemizeminus}
\end{itemizedot}


\tmtextbf{Human Computer Interaction / User Interfaces:} Human computer
interaction is the study of interaction between people (users) and computers.
It is an interdisciplinary subject, relating computer science with many other
fields of study and research. Interaction between users and computers occurs
at the user interface.
\begin{itemizedot}
  \item Own Research: Development of user interfaces for the mathematical
  assistance environment OMEGA. Our goal is to support human-oriented user
  interaction in OMEGA, ideally based on mixture of natural language input,
  formulas, and graphics. A further aim is to integrate the user interface of
  OMEGA with the scientific editing platform TeXmacs.
  
  \item Selected Publications:
  \begin{itemizeminus}
    \item [J10] S. Autexier, C. Benzm\"uller, A. Fiedler, H. Lesourd,
    \tmtextbf{Integrating Proof Assistants as Reasoning and Verification Tools
    into a Scientific WYSIWYG Editor.} \tmtextbf{
    }\href{http://www.elsevier.com/wps/find/journaldescription.cws\_home/681021/description\#description}{Electronic
    Notes in Theoretical Computer Science}. Elsevier. To appear, 2006.
    
    \item [J2] {\small J. Siekmann, S. Hess, C. Benzm\"uller, L. Cheikhrouhou,
    A. Fiedler, H. Horacek, M. Kohlhase, K. Konrad, A. Meier, E. Melis, M.
    Pollet, and V. \ Sorge, \tmtextbf{LOUI: Lovely Omega User Interface.} }
    {\small
    \href{http://www.springerlink.com/app/home/journal.asp?wasp=e2wqmq6uqn4kpwexcd4p\&referrer=parent\&backto=linkingpublicationresults,id:102822,1}{Formal
    Aspects of Computing}, (1999) 11:326--342. Springer.}
  \end{itemizeminus}
\end{itemizedot}


\tmtextbf{Maschine Learning:} Machine learning is concerned with the
development of algorithms and techniques, which allow computers to "learn". At
a general level, there are two types of learning: inductive, and deductive.
Inductive machine learning methods create computer programs by extracting
rules and patterns out of massive data sets.
\begin{itemizedot}
  \item Own Research: Learning of proof methods in mathematics and their
  automatic encoding as proof methods in OMEGA.
  
  \item Selected Publications:
  \begin{itemizeminus}
    \item [J7] {\small M. Jamnik, M. Kerber, M. Pollet, and C. Benzm\"uller,
    \tmtextbf{Automatic Learning of Proof Methods in Proof Planning. }}{\small
    \href{http://www3.oup.co.uk/igpl/}{The Logic Journal of the IGPL}, (2003)
    11(6):}{\small 647-674. Oxford University Press.}
  \end{itemizeminus}
\end{itemizedot}


\tmtextbf{{\tmabbr{Intelligent Tutor System}}s:} An intelligent tutoring
system is any computer system that provides direct - i.e. without the
intervention of human beings - customised feedback to students.
\begin{itemizedot}
  \item Own Research: In the SFB378 DIALOG project we investigate whether the
  tutoring of mathematical proofs can be fruitfully supported by reasoning
  techniques provided in the mathematical assistance environment OMEGA. We
  combine empirical studies, by which we obtain corpora for analysis, with
  system modeling and system implementation.
  
  \item Selected Publications:
  \begin{itemizeminus}
    \item [C22] C. Benzm\"uller, H. Horacek, H. Lesourd, I. Kruijff-Korbayova,
    M. Schiller, M. Wolska, \tmtextbf{DiaWOz-II - A Tool for Wizard-of-Oz
    Experiments in Mathematics.} KI 2006: Advances in Artificial Intelligence:
    29th Annual German Conference on AI, LNAI, Bremen, Germany, 2006.
    Springer. To appear.
    
    \item [C20] C. Benzm\"uller, H. Horacek, H. Lesourd, I. Kruijff-Korbayova,
    M. Schiller, and M. Wolska, \tmtextbf{A corpus of tutorial dialogs on
    theorem proving; the influence of the presentation of the study-material.}
    \ Proceedings of International Conference on Language Resources and
    Evaluation (LREC), Genova, Italy, 2006. ELDA.
    
    \item [C18] C. Benzm\"uller and Q.B. Vo, \tmtextbf{Mathematical Domain
    Reasoning Tasks in Tutorial Natural Language Dialog on Proofs}.
    Proceedings of the Twentieth National Conference on Artificial
    Intelligence (AAAI-05), pp. 516-522, Pittsburgh, Pennsylvania, 2005. USA.
    AAAI Press / The MIT Press.
  \end{itemizeminus}
\end{itemizedot}


\tmtextbf{{\tmabbr{Dialog Systems}} and \tmtextbf{Natural Language
Processing}:} A Dialog system is a computer system intended to converse with a
human. There are many different architectures for dialog systems. Principle to
any dialog system is the dialog manager, which is a component or set of
components that manages the state of the dialog. Natural language processing
is a subfield of artificial intelligence and linguistics. It studies the
problems of automated generation and understanding of natural human languages.
\begin{itemizedot}
  \item Own Research: Development of a natural language-based tutorial DIALOG
  system in the SFB378.
  
  \item Selected Publications:
  \begin{itemizeminus}
    \item [J14] C. Benzm\"uller, H. Horacek, I. Kruijff-Korbayova, M. Pinkal,
    J. Siekmann, M. Wolska, \ \tmtextbf{Natural Language Dialog with a Tutor
    System for Mathematical Proofs}. \ \href{http://jcst.ict.ac.cn/}{Journal
    of Computer Science and Technology}. To appear, 2006.
    
    \item [C21] M. Buckley and C. Benzm\"uller, \tmtextbf{An Agent-based
    Architecture for Dialogue Systems.} Sixth International Andrei Ershov
    Memorial Conference 'Perspectives of System Informatics' (PSI'06),
    Novosibirsk, Akademgorodok, Russia, 2006. Springer LNCS. To appear.
  \end{itemizeminus}
\end{itemizedot}






{\noindent}\begin{tabular}{l}
  \hline
  \quad
\end{tabular}

\tmtextbf{\begin{center}
  Theoretical Computer Science
\end{center}}



\tmtextbf{Proof Theory}: Proof theory is a branch of mathematical logic that
represents proofs as formal mathematical objects, facilitating their analysis
by mathematical techniques. Proof theory is syntactic in nature, in contrast
to model theory, which is semantic in nature. Together with model theory,
axiomatic set theory, and recursion theory, proof theory is one of the
so-called four pillars of the foundations of mathematics.
\begin{itemizedot}
  \item Own Research: Proof theory in higher-order logic with a particular
  focus on extensionality and cut.
  
  \item Selected Publications:
  \begin{itemizeminus}
    \item [C23] C. Benzm\"uller, C. Brown, and M. Kohlhase,
    \tmtextbf{Cut-Simulation in Impredicative Logics.} Third International
    Joint Conference on Automated Reasoning (IJCAR'06), LNAI, Seattle, USA,
    2006. Springer. To appear.
    
    \item [J5] {\small C. Benzm\"uller, \tmtextbf{Comparing Approaches to
    Resolution based Higher-Order Theorem Proving.}}{\small 
    \href{http://www.kluweronline.com/issn/0039-7857}{Synthese}, (2002)
    133(1-2):203-235. Kluwer.}
    
    \item [T2] {\small C. Benzm\"uller, \tmtextbf{Equality and Extensionality
    in Higher-Order Theorem Proving. }}{\small Doctoral Thesis, Computer
    Science, Saarland University, April 1999.}
    
    \item {\small [C2] C. Benzm\"uller and M. Kohlhase, \tmtextbf{Extensional
    Higher-Order Resolution.}} {\small In} {\small C. Kirchner and H. Kirchner
    (eds.)}{\small , Proceedings of the 15th International Conference on
    Automated Deduction (CADE), \href{http://www.springer.de/comp/lncs/}{LNAI}
    vol. 1421, pp. 56-71, Lindau, Germany, 1998. Springer.}
  \end{itemizeminus}
\end{itemizedot}


\tmtextbf{Semantics and Model Theory:} Many of the formal approaches to
semantics applied in linguistics, mathematical logic, and computer science
originated in techniques for the semantics of logic, most influentially being
Alfred Tarski's ideas in model theory and his semantic theory of truth. In
mathematics, model theory is the study of the representation of mathematical
concepts in terms of set theory, or the study of the models which underlie
mathematical systems.
\begin{itemizedot}
  \item Own Research: Semantics and model theory in higher-order logics. A
  particular contribution is the development of different notions of model
  classes which generalize Henkin Semantics. Since these model classes do not
  prerequire functional and Boolean extensionality they are of interest, for
  example, for program verification (we may want to distinguish an inefficient
  program from an efficient one even if both programs have exactly the same
  input-/output-behavior) and in computational linguistics or multi-agent
  system semantics \ (where intensional concepts such as knowledge and belief
  are relevant). \
  
  \item Selected Publications:
  \begin{itemizeminus}
    \item [C17] C. Benzm\"uller and C. Brown, \tmtextbf{A Structured Set of
    Higher-Order Problems}. Proceedings of the 18th International Conference
    on Theorem Proving in Higher Order Logics (TPHOLs 2005),
    \href{http://www.springer.de/comp/lncs/}{LNAI} vol. 3606, pp. 66-81,
    Oxford, UK, 2005. Springer.
    
    \item [J6] {\small C. Benzm\"uller, C. Brown, and M. Kohlhase,
    \tmtextbf{Higher Order Semantics and Extensionality.} }{\small
    \href{http://www.jstor.org/journals/00224812.html}{Journal of Symbolic
    Logic}. (2004) 69(4):1027-1088.}{\small  JSTOR.}
  \end{itemizeminus}
\end{itemizedot}


\tmtextbf{Formal Methods:} Formal methods refers to mathematically based
techniques for the specification, development and verification of software and
hardware systems. The approach is especially important in high-integrity
systems, for example where safety or security is important, to help ensure
that errors are not introduced into the development process.
\begin{itemizedot}
  \item Own Research: Starting in October 2006 I will develop the higher-order
  theorem LEO-II at University of Cambridge in cooperation with Larry Paulson
  (EPSRC grant EP/D070511/1). The prover will be designed to solve problems of
  the sort that arise in verification. One design goal is that it should be
  easy to integrate with interactive verification tools such as HOL and
  Isabelle.
  
  My masters thesis work in 1994 did also focus on formal methods.
  
  \item Selected Publications:
  \begin{itemizeminus}
    \item no publications yet on the LEO-II project; see the original proposal
    at \href{http://www.cl.cam.ac.uk/\~{
    }lp15/Grants/leo2.html}{http://www.cl.cam.ac.uk/\~{
    }lp15/Grants/leo2.html} for more details.
    
    \item [T1] {\small C. Benzm\"uller, \tmtextbf{Eine Fallstudie zur
    Spezifikation von Systemanforderungen in der Spezifikationssprache
    OBSCURE.}} Masters Thesis, \ Saarland University, Saarbr\"ucken, 1994.
  \end{itemizeminus}
\end{itemizedot}


\tmtextbf{Functional Programming:} Functional programming is a programming
paradigm that treats computation as the evaluation of mathematical functions.
Functional programming emphasizes the definition of functions rather than the
implementation of state machines, in contrast to procedural programming, which
emphasizes the execution of sequential commands.
\begin{itemizedot}
  \item Own Research: Though I am not pursuing any direct research in
  functional programming, I have a strong background in this area since all
  the above systems are implemented in functional programming languages.
  Furthermore, typed lambda calculi play a fundamental role for functional
  programming and these are in turn subject of my research on proof theory and
  model theory in higher-order logic. 
\end{itemizedot}}}



\end{document}
