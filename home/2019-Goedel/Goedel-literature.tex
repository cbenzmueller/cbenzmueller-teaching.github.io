\documentclass[12pt]{article}  
\usepackage[T1]{fontenc}
\usepackage{hyperref,a4wide}

\usepackage{natbib}

\newcommand{\SortNoop}[1]{}

   	\title{Literature suggestion for the G\" odel seminar}
        \author{Sre\'{c}ko Kova\v{c} and Christoph Benzm\"uller}


	\date{ }
	
\begin{document}
\maketitle
\thispagestyle{empty}

%(Remark: Further suggestions will be added in due time) \\[2em]

\begin{enumerate}
\item Concise overview on Kurt Gödel and his writings:
  \cite{sep-goedel}


\item  G\" odel's general philosophical views: \cite{goed:russell},
  \cite{goed:basic},  \cite{goed:modern},
  \cite{goed:cantor-orig-64}. Some more explicit sources:
  \cite{wang:96}, and \cite{crocco:hal-01459188}.
  (\cite{crocco:KGP-S} contains many extracts from G\" odel's
  unpublished notes.) A good overview is provided in \cite{atten:03}.



\item {Foundations and philosophy of mathematics:} intuition,
  completeness theorem, computability and analyticity, set-theoretic
  multiverse; see chapters in \cite{kennedy14}.

\item Ontological argument:\cite{goed:onto}. \cite{goed:kant} is also
  helpful. 

\item Attempts to formalize provability and the distinction of the concept of ``absolute proof'' and a proof in a concrete formal system: \cite{goed:ext}

\item On connections between provability and Turing machine:
  \cite{goed:bicen}, \cite[postscriptum on
  pp. 369--371]{goed:undec-prop}, also in
  \cite[p. 308--309]{goed:basic}, and in \cite[note on
  p. 195]{goed:undec}, \cite{goed:remarks}.

\item Connection between intuitionistic and classical logic:
  \cite{Godel1933,Godel1969-GDEAIO}

\item Notion of time: \cite{godel9:_einst,goed:basic}; maybe also
  \cite{goldstein06:_incom,Yourgrau1991-YOUTDO-4}


\item \cite{godel30:_volls_axiom_funkt,goed:undec} are general pretext
  for the above topics; see also \cite{nagel01} for a non-technical
  introduction.

\item Gödel's first steps in Logik: \cite{vanPlatoBSL18}

\item Recent article on Gödel's incompleteness theorem: \cite{vanPlatoICM18}

 \end{enumerate}



\bibliographystyle{plainnat}
\bibliography{Goedel-literature}

\end{document}




